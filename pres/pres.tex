\documentclass{beamer}
\usetheme{CambridgeUS}
\usepackage[absolute,overlay]{textpos}
\usepackage{graphicx}
\usepackage{tikz}
\usepackage{animate}
\usepackage{calc}
\newcounter{m} % Number of sides of the polygon
\setcounter{m}{2}
\newcounter{c} % Internal Control Counter
\usepackage{hyperref}
\hypersetup{pdfpagemode=FullScreen}
\usefonttheme{professionalfonts}
\usepackage{times}
\usepackage{changepage}
\providecommand\thispdfpagelabel[1]{}

\usepackage{amssymb}
\usepackage{tikz}
\usepackage{amsmath}
\usepackage{verbatim}
\usetikzlibrary{arrows,shapes}
\usepackage{array,multirow,makecell}
\setcellgapes{1pt}
\makegapedcells
\newcolumntype{R}[1]{>{\raggedleft\arraybackslash }b{#1}}
\newcolumntype{L}[1]{>{\raggedright\arraybackslash }b{#1}}
\newcolumntype{C}[1]{>{\centering\arraybackslash }b{#1}}
\tikzset{
pil/.style={
           ->,
           thick,
           shorten <=2pt,
           shorten >=2pt,}}
\usetikzlibrary{arrows,shapes}
\usetikzlibrary{arrows,positioning}
\usepackage{smartdiagram}
\usepackage{metalogo}
\usepackage{dtklogos}
\setbeamertemplate{headline}{}
\tikzstyle{vecArrow} = [thick, decoration={markings,mark=at position
   1 with {\arrow[semithick]{open triangle 60}}},
   double distance=1.4pt, shorten >= 5.5pt,
   preaction = {decorate},
   postaction = {draw,line width=1.4pt, white,shorten >= 4.5pt}]
\tikzstyle{innerWhite} = [semithick, white,line width=1.4pt, shorten >= 4.5pt]

\setbeamertemplate{navigation symbols}{}
\makeatletter
\setbeamertemplate{footline}
{
  \leavevmode%
  \hbox{%
  \begin{beamercolorbox}[wd=.333333\paperwidth,ht=2.25ex,dp=1ex,center]{author in head/foot}%
    \usebeamerfont{author in head/foot} {Onur, Bastien, Paul, Florent}
  \end{beamercolorbox}%
  \begin{beamercolorbox}[wd=.333333\paperwidth,ht=2.25ex,dp=1ex,center]{title in head/foot}%
    \usebeamerfont{title in head/foot}\insertsection
  \end{beamercolorbox}%
  \begin{beamercolorbox}[wd=.333333\paperwidth,ht=2.25ex,dp=1ex,right]{date in head/foot}%
    \usebeamerfont{date in head/foot}\insertshortdate{}\hspace*{2em}
    \insertframenumber{} / \inserttotalframenumber\hspace*{2ex} 
  \end{beamercolorbox}}%
  \vskip0pt%
}
\makeatother

\newcommand{\margin}{0\paperwidth}


%\beamersetrightmargin{\margin}
%\beamersetleftmargin{\margin}

\newcommand\Wider[2][3em]{%
\makebox[\linewidth][c]{%
  \begin{minipage}{\dimexpr\textwidth+#1\relax}
  \raggedright#2
  \end{minipage}%
  }%
}



\usetikzlibrary{shapes,snakes}

%\input{tex_files/ZZ_newcommand1}

\begin{document}

%\usepackage{framed}
%\usepackage{cite,url}
%\usepackage{amsmath,amssymb}
%\usepackage{amscd}
%\usepackage[english] {babel}
%\usepackage[pdftex]{graphicx,color}
%\usepackage{stmaryrd}
%\usepackage{amsmath}
%\usepackage{xspace,calc}
%\usepackage{games}
%\usepackage{booktabs}
%\usepackage{needspace}
%\use package{times}
%\usepackage{algorithmicx} % sorry

\bibliography{abbrev3,crypto,add,add2}


\title{Using NetZob for Active protocol reverse engineering }   
\author{Onur Catakoglu , Bastien Drouot, Paul Germouty, Florent Tardif} 
\date{November 2, 2017} 


%\logo{%
%    \includegraphics[width=1cm,height=0.5cm,keepaspectratio]{limoges.jpg}~%
%    \includegraphics[width=1cm,height=0.5cm,keepaspectratio]{paris2logo.jpg}%
%}

%\logo{\includegraphics[height=5mm]{limoges.jpg}}

\begin{frame}

	\titlepage
\end{frame}


%%%%%%%%%%%%%%%%%%%%%%%%%%%%%%%%%%%%%%%%%%



\begin{frame}


	\tableofcontents

\end{frame}

\section{Introduction}
\begin{frame}

	\tableofcontents[currentsection]
\end{frame}

\begin{frame}{Objectives-Attack Model}
\begin{itemize}
\item Reverse engineering transmission protocols

$\Rightarrow$ characterize protocols' structure

\item Client-owning Model

$\Rightarrow$ possibility to discuss with the server with the protocol
\end{itemize}
\end{frame}

\begin{frame}{Previous Work \textbf{Netzob}}
Passive framework for analysing transmission
\begin{itemize}
\item symbol
\item \textit{Format.splitAligned(s)}
\item \textit{Format.splitStatic(s)}
\item \textit{Format.splitDelimiter(s,Raw($listofbytes$))}
\end{itemize}


\end{frame}

\begin{frame}{Exemples}
symbol screen shot
\end{frame}

\begin{frame}{Exemples}
Aligned screen shot
\end{frame}

\begin{frame}{Exemples}
Static screen shot
\end{frame}

\begin{frame}{Exemples}
Delimiter screen shot
\end{frame}

\section{RoadMap}
\begin{frame}

	\tableofcontents[currentsection]
\end{frame}

\begin{frame}{Road Map of GoodGuy ReverseIngBoy}

\end{frame}
\begin{frame}{Getting Information}
getting pcap
\end{frame}

\begin{frame}{Divide and Conquer}


sorting client server

sorting type of request (s7: co, rm, wm, dc)

recognize field (delimiters approach, Shaping to split aligns)
\end{frame}

\begin{frame}{Analysing Shape}

\end{frame}

\begin{frame}
IDK what is left i guess
\end{frame}
%\section{Basic Tools}
%\begin{frame}
%
%	\tableofcontents[currentsection]
%\end{frame}
%
%\begin{frame}
%getting pcap
%
%sorting pcap
%
%verify cruciality
%
%verify timeout
%
%send 1 message
%...
%
%
%\end{frame}
\section{Reconstructing Fields}
\begin{frame}

	\tableofcontents[currentsection]
\end{frame}

\begin{frame}{getting pcap and sorting}

\end{frame}

%
\begin{frame}
delimiters

shaping to split

\end{frame}

\section {Characterizing Fields}
\begin{frame}

	\tableofcontents[currentsection]
\end{frame}

\begin{frame}
static/dyn?

bin/text?

nb
\end{frame}


\section{Finding Protocol Behaviour}
\begin{frame}

	\tableofcontents[currentsection]
\end{frame}

\begin{frame}
Stateless, relations
\end{frame}

\section{Experiments demonstration}
\begin{frame}

	\tableofcontents[currentsection]
\end{frame}

\section{Future Work}
\begin{frame}

	\tableofcontents[currentsection]
\end{frame}


\end{document}